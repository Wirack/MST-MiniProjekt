% Standard arkitektur
% Alternativer
% Nogle der takler security
%Arkitekturer der forsimpler kommunikation
% Openmobster - http://ieeexplore.ieee.org.zorac.aub.aau.dk/stamp/stamp.jsp?tp=&arnumber=6834971
We will provide an overview of established MCC architectures, such as Centralized Cloud, Cloudlet, Ad Hoc Mobile, as well as open architectural challenges.
\begin{description}[style=nextline]
\item [Centralized Cloud]
This type of architecture places the cloud resource in a remote centralized cloud infrastructure\cite[p.3]{liu2013gearing}.Mobile devices acces data center resources by using the network via WIFI, og 3G/4G. The cloud is an agent between original content providers and mobile devices. Mobile devices offload parts of their application workload to the cloud.
\item [Cloudlet]
Is a well connected, resource-rich server\cite[p.3-4]{liu2013gearing}. Used similarly to WiFi hotspots to get low latency to the users, sp it does not have the long latency delays of centralized clouds. However due to having higher cost per area covered, it is sparsely deployed.
\item [Ad Hoc Mobile Cloud]
When the cloud or cloudlet is not available or affordable, an Ad Hoc Mobile Cloud can be used to pool together nearby mobile devices for resource sharing\cite[p.4]{liu2013gearing}. The workload is distributed among the crowd of devices by distributing or collaborating on tasks. 
\item [Architectural Challenges]
Creating a reference architecture for the heterogenous MCC environment can utilize the great potential of ubiquitous mobile computing\cite{sanaei2014heterogeneity}. This may be achieved by using a combination of arcitechtures. Cloudlets can be placed in hot spots between mobile users and data centers for latency-sensitive applications. Ad hoc mobile cloud can be used as a backup solution when a group of users, like subway passengers, can not access the cloudlets or data centers. Additionally, the individual Ad hoc devices with stable Internet connection can utilize their WiFi hotspot capability as a relay for nearby poorly connected devices.



%\cite{article:mobilecloudcomputingarchkilde}

\end{description}