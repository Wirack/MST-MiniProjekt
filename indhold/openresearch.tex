In the context of MCC multiple research areas exist, where some are described hereafter.
\subsection{Energy Saving}\label{EnergySaving}
%Energy Saving - Bad connection = More energy - Small computations + Cloud = More Energy
When having to make computations on a mobile phone it can be very energy consuming.
By using MCC, you might think that you can reduce this.
However, this is not always the case.
One case where it can be much more energy consuming to calculate on the cloud is when you are making low cost computations \citep{goyalmobile,liu2013gearing,fernando2013mobile}.
The reason why this is much more energy consuming is because it cost a lot of energy to send data through the internet, but also to receive new data.
Furthermore, if you are having a bad internet connection it is also more energy consuming to send and receive data \citep{goyalmobile,liu2013gearing,fernando2013mobile}.
It is therefore important to consider when it is profitable to use MCC before sending data through the internet.

\subsection{Privacy}
%Privacy
Due to mobile devices usually being used as a personal device, privacy is an important factor for the users of a MCC system.
As \citet{liu2013gearing} and \citet{fernando2013mobile} describes it, many users have private information on their mobile devices, which the users does not want to be spread to all of the internet.
Therefore, it is important that the users are able to see and or decide what data are shared with the MCC system, as mentioned in \citet{sanaei2014heterogeneity}.
However, sometimes the MCC system must have access to a certain type of data, and in this case the user would have to trust the system \citep{fernando2013mobile}.

\subsection{Quality of Internet Connection}
%Quality of communication
As mentioned in section \ref{EnergySaving} the quality of the internet connection can be an important factor.
However, this quality also have other consequences, such as connection latency.
As mentioned by \citet{fernando2013mobile} the delay can be as high as 680ms on a 3G network.
The reason why this delay is so high is because the mobile device is far away from the internet provider, but also because the connection is wireless.
When connecting through a wireless connection many things can have an impact, e.g. the weather \citep{dev2014review,kumar2013mobile}.
The delay can have a negative impact on the experience when using an application, because it can be very frustrating for the user to wait a long time on a result \citep{hazarika2014mobile}.

\subsection{Context-Awareness}
As discussed in \citet{5557960}, a current research topic is to use the sensors of mobile devices to determine the context they are in.
This way they can provide context awareness and thus the service in the cloud could be personalized based on the context, e.g as \citet{fernando2013mobile} says if the context changes that might mean a different cloud service would be used for that. 
Given a context for a user, such as a combination of their location and their movement you could provide different cloud services based on those contexts. 
This could be as \citet{5557960} suggests, a location-specific map service. 
A more concrete example would be a map service of the building the user is currently in \citep{sanaei2014heterogeneity}.
