As with everything else involving technology, MCC have some security issues.
These security issues are on the mobile device, during network communication and while on the cloud.
\citet{6923101} presents a list of security issues in all three of these categories among these are on the cloud side malicious content injection (ex data received from a virus infected phone), data integrity (due to the divided storage of cloud data) or Distributed Denial of Service (DDoS) that attempts to bring down the cloud service.
The communication channel has the issue of it being possible to intercept data in transit between phone and cloud, resulting in data leaks, and also a risk of data being altered before it reaches the cloud.
On the side of the mobile devices security issues include things such as unauthorized access (in case of stolen phone) and tampering with the application by malicious software on the phone.
Potential solutions to some of these issues are presented in \citet{6583635}.
Among these are a suggested solution to deal with malicious software on the phone.
Here the suggested solution is to use cloud computational capabilities to detect malicious software on the phone through the program called CloudAV.
In the case of malware detection by CloudAV \citep{Oberheide:2008:CNA:1496711.1496718}, it sends the necessary tools to the phone to remove it, which makes it so that the phone only need to use processing power to actually remove the threat, rather than detecting it as well.

However, these are of course not the only potential solutions, \citet{7056876} suggest various hardware-based approaches through trusted platform solutions to secure cloud data and network communication.
These trusted platforms use secure hardware elements to ensure that only the appropriate processes access the cloud computational area.
Whether or not this approach is a good solution is uncertain, as this is still very much an open research topic, however, one security problem it is not likely to solve is the risk of DDoS attacks to bring down cloud services.

%http://ieeexplore.ieee.org.zorac.aub.aau.dk/stamp/stamp.jsp?tp=&arnumber=6996177
%http://ieeexplore.ieee.org.zorac.aub.aau.dk/stamp/stamp.jsp?tp=&arnumber=6732435
%http://ieeexplore.ieee.org.zorac.aub.aau.dk/stamp/stamp.jsp?tp=&arnumber=6047291
%http://ieeexplore.ieee.org.zorac.aub.aau.dk/stamp/stamp.jsp?tp=&arnumber=6688818