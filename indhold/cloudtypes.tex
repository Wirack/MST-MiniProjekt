When it comes to the description of clouds, people distinguish between four types of clouds \citep{article:mobilecloudreviewinderkilde,article:mobilecloudreviewinderkildesecurity}.
\begin{description} [style=nextline]
\item[Private Cloud]
	Is a cloud kept private for security reason, usually used internally in an organisation \citep{article:mobilecloudreviewinderkildesecurity}.
	In such a cloud users pay per use, but for this payment they get increased security and privacy \citep{article:mobilecloudreviewinderkilde}.
	A common example of such a cloud is Amazon Virtual Private Cloud \citep{article:mobilecloudreviewinderkilde}.
\item[Public Cloud]
	Is a cloud that is open for the public to use. It can be educational, governmental or for personal use. Services can be free or users have to pay depending on the service they subscribe to \citep{article:mobilecloudreviewinderkilde}.
	Common examples are Google Drive \citep{article:mobilecloudreviewinderkildesecurity}, Google App Engine, and Amazon Elastic Cloud Compute \citep{article:mobilecloudreviewinderkilde}. 
\item[Hybrid Cloud]
	This is a combination of the private and public cloud.
	The main idea is that you keep the cloud private at first, but if there is a demand for higher computing power, you delegate some of the work to a public sector \citep{article:mobilecloudreviewinderkilde}. With this in mind, it is crucial that it is the non-sensitive data that is computed in the public cloud.
\item[Community Cloud]
	This fourth type governs the use of several firms and institutions with common interests \citep{article:mobilecloudreviewinderkilde}.
	An example of this is an education or medical community \citep{article:mobilecloudreviewinderkildesecurity}.
	The idea is then that you share the cloud with other stakeholders having common interests.
\end{description}
These types of clouds touches upon the security aspects of accessibility.
This may vary a lot, depending on the context and application being developed.
Do the customer work with highly sensitive data or not, is only some data sensitive, or does the customer want to restrict access to a certain community.

How these clouds are utilized in a sensible manner is another discussion that is described later in this survey. That discussion goes into various frameworks that can be used as well as common architectures for MCC.
